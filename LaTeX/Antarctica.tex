\documentclass{article}
\usepackage{url}
\usepackage{hyperref}
\usepackage{gentium}
\usepackage{multicol}
\usepackage[margin=2cm]{geometry}

\begin{document}

\section{Ocean Conservancy}
Ocean Conservancy is a nonprofit advocacy group involved with governments around the globe 

\begin{multicols}{2}[
    \section{Marine plastics}
]
\begin{itemize}
    \item Two main groups of sources for microplastics have been identified: primary microplastics are those which enter the ocean directly, coming from sources such as toothpaste, laundry products, etc. Secondary microplastics come from degradation of macroplastics in the ocean. Many of these secondary sources float on the surface of the ocean and are subject to exposure to UV radiation \cite{microplastics2017}.
    
    \item Primary microplastics found in cosmetics which are poured down the drain cannot be collected and recycled (unlike macroplastics), and do not decompose in wastewater treatment plants \cite{unepcosmetics2015}.
    
    \item All regions of the world contribute a significant amount of microplastics pollution from cosmetics \cite{pmpsinoceans2017}.
    
    \item There has been action from western governments to reduce the presence of microplastics in cosmetics \cite{mpsincosmetics2019}. Furthermore, the European Commission has communicated a strategy to make all plastics in the EU reusable or recyclable by 2030 \cite{ecplastics2018}.
    
    \item Organizations across the Antarctic, including the Commission for the Conservation of Antarctic Marine Living Resources (CCAMLR), have recently taken steps to look at microplastic concentrations around areas of high human activity (\url{https://www.ccamlr.org/en/sc-camlr-xxxvi/bg/29}). However, CCAMLR is focused on the marine environment and not on the interior of the Antarctic.
    
    \item The marine environment is 
\end{itemize}
\end{multicols}

\begin{multicols}{2}[
    \section{Land debris}
]
\end{multicols}

\begin{thebibliography}{1}
    \bibitem{microplastics2017}
        ``Microplastics in the Antarctic marine system: An emerging area of research'',
        \url{http://dx.doi.org/10.1016/j.scitotenv.2017.03.283}; 2017

    \bibitem{unepcosmetics2015}
        ``Plastics in Cosmetics'', \
        \url{http://hdl.handle.net/20.500.11822/9664}; 2015

    \bibitem{pmpsinoceans2017}
        ``Primary microplastics in the oceans'', 
        \url{https://www.iucn.org/content/primary-microplastics-oceans}; 2017

    \bibitem{mpsincosmetics2019}
        ``Microplastics in cosmetics: Environmental issues and needs for global bans'', 
        \url{https://www.sciencedirect.com/science/article/pii/S1382668918305635}; 2019

    \bibitem{ecplastics2018}
        ``A European Strategy for Plastics in a Circular Economy'', 
        \url{https://eur-lex.europa.eu/legal-content/EN/TXT/?uri=CELEX:52018DC0028}; 2018
\end{thebibliography}


\end{document}