\documentclass{article}
\usepackage{url}
\usepackage{hyperref}
\usepackage{gentium}
\usepackage{multicol}
\usepackage[margin=2cm]{geometry}
\usepackage{tikzsymbols}

\begin{document}

\begin{multicols}{2}[\section{Organizations involved}]
\begin{itemize}
    \item Ocean Conservancy (OC) is a nonprofit advocacy group involved with governments around the globe to formulate policies protecting ocean ecosystems. They have a number of programs in place (available at https://oceanconservancy.org/programs/). Important among these with regards to the Antarctic sabbatical include the Trash Free Seas program. Within this, OC organizes the annual International Coastal Cleanups, bringing together millions of volunteers to clear up millions of tonnes of plastic trash on beaches across the world. OC works with individuals, businesses, and governments to bring awareness to the problem with plastic, and to prevent it from entering the ocean altogether. Also, I learned of a new bird from one of their factsheets: (\url{https://oceanconservancy.org/wildlife-factsheet/laysan-albatross/})! They also organize programs protecting the Arctic, Florida, and the oceanic climate as a whole.
    
    \item The British Antarctic Survey (BAS) is the United Kingdom's polar research arm, a component of the National Environment Research Council (NERC), which is in turn a part of UK Research and Innovation (UKRI). How deep does the research hole go? With an annual budget of £50 million, the BAS allocates this (along with income tax from the British Antarctic territory, funding from the EU, and other sources) to a variety of scientific activities. On their website they maintain a list of research projects (\url{https://www.bas.ac.uk/science/our-research/research-projects/}) as well as research topics in general (\url{https://www.bas.ac.uk/science/our-research/topics/}). Recently, the RRS \textit{Sir David Attenborough}, a new polar ship operated by the BAS was formally named (\url{https://www.bas.ac.uk/media-post/ship-is-named-with-royal-ceremony/}). Carried on the vessel is the lead boat of a class of autonomous underwater vehicles, \textit{Boaty McBoatface}. Curiously, as I was reading about the naming ceremony of the new vessel, I came across plastic-eating enzyme research. Hopefully it'll go far!
    
    \item Antarctic Logistics and Expeditions (ALE) do precisely what their name suggests - they provide expeditions to Antarctica and organize logistics for making it happen.
    
    \item Airbnb is a company arranging and offering lodging and tourism experiences around the world. They act as a broker, meaning this sabbatical is hosted by ALE, with Airbnb helping put citizens and scientists together.

\end{itemize}
\end{multicols}

\begin{multicols}{2}[
    \section{Marine plastics}
]
\begin{itemize}
    \item Two main groups of sources for microplastics have been identified: primary microplastics are those which enter the ocean directly, coming from sources such as toothpaste, laundry products, etc. Secondary microplastics come from degradation of macroplastics in the ocean. Many of these secondary sources float on the surface of the ocean and are subject to exposure to UV radiation \cite{microplastics2017}.
    
    \item Primary microplastics found in cosmetics which are poured down the drain cannot be collected and recycled (unlike macroplastics), and do not decompose in wastewater treatment plants \cite{unepcosmetics2015}.
    
    \item All regions of the world contribute a significant amount of microplastics pollution from cosmetics \cite{pmpsinoceans2017}.
    
    \item There has been action from western governments to reduce the presence of microplastics in cosmetics \cite{mpsincosmetics2019}. Furthermore, the European Commission has communicated a strategy to make all plastics in the EU reusable or recyclable by 2030 \cite{ecplastics2018}.
    
    \item Organizations across the Antarctic, including the Commission for the Conservation of Antarctic Marine Living Resources (CCAMLR), have recently taken steps to look at microplastic concentrations around areas of high human activity (\url{https://www.ccamlr.org/en/sc-camlr-xxxvi/bg/29}). However, CCAMLR is focused on the marine environment and not on the interior of the Antarctic.
    
    \item The marine environment is 
\end{itemize}
\end{multicols}

\begin{multicols}{2}[
    \section{Land debris}
]
\end{multicols}

\begin{thebibliography}{1}
    \bibitem{microplastics2017}
        ``Microplastics in the Antarctic marine system: An emerging area of research'',
        \url{http://dx.doi.org/10.1016/j.scitotenv.2017.03.283}; 2017

    \bibitem{unepcosmetics2015}
        ``Plastics in Cosmetics'', \
        \url{http://hdl.handle.net/20.500.11822/9664}; 2015

    \bibitem{pmpsinoceans2017}
        ``Primary microplastics in the oceans'', 
        \url{https://www.iucn.org/content/primary-microplastics-oceans}; 2017

    \bibitem{mpsincosmetics2019}
        ``Microplastics in cosmetics: Environmental issues and needs for global bans'', 
        \url{https://www.sciencedirect.com/science/article/pii/S1382668918305635}; 2019

    \bibitem{ecplastics2018}
        ``A European Strategy for Plastics in a Circular Economy'', 
        \url{https://eur-lex.europa.eu/legal-content/EN/TXT/?uri=CELEX:52018DC0028}; 2018
\end{thebibliography}


\end{document}